\documentclass{article}
\usepackage{ctex}  % 支持中文
\usepackage{geometry}
\usepackage{graphicx}
\usepackage{float}
\usepackage{longtable}
\usepackage{booktabs}
\usepackage{array}
\usepackage{multirow}
\usepackage{xcolor}
\usepackage{colortbl}
\usepackage{enumitem}
\usepackage{hyperref}

\geometry{a4paper, left=2.5cm, right=2.5cm, top=2.5cm, bottom=2.5cm}

\begin{document}
	\begin{center}
		{\LARGE 高铁订票管理系统软件测试报告}
	\end{center}
	
	\vspace{1cm}
	
	\begin{tabular}{rl}
		测试项目: & 高铁订票管理系统 \\
		班级: & 信息工程学院计算机科学与技术一班 \\
		团队成员:& 王宇豪(1231001024)、熊江伟(1231001026)\\
		测试日期: & 2025年12月23日 \\
		测试环境: & Windows 10/11, Qt 6.x
	\end{tabular}
	\\
	\\
	\\
	\tableofcontents
	\newpage
	
	\section{测试概述}
		\subsection{测试目的}
			本测试报告旨在对高铁订票管理系统进行全面的黑盒测试,验证系统各项功能的正确性、稳定性和用户体验。测试重点关注用户界面交互、业务逻辑处理、数据完整性以及系统性能等方面,确保系统满足设计要求并能够正常运行。
		
		\subsection{测试范围}
			测试范围涵盖以下模块:
			\begin{itemize}[leftmargin=2cm]
				\item 用户端功能:用户登录/注册、余票查询、订单管理、乘车人管理、个人信息管理
				\item 管理员功能:管理员登录、车次管理、订单管理、用户管理
				\item 通用功能:时刻表查看、数据持久化、系统启动与退出
				\item 非功能性测试:内存泄漏、CPU占用、界面响应速度
			\end{itemize}
		
		\subsection{测试环境}
			\begin{table}[H]
				\centering
				\begin{tabular}{|c|c|}
					\hline
					\textbf{测试项} & \textbf{配置信息} \\ \hline
					操作系统 & Windows 11 / Windows 10 \\ \hline
					开发工具 & Qt Creator 14.x / Qt 6.x \\ \hline
					编译器 & MinGW 13.1.0 / MSVC 2022 \\ \hline
					CPU & AMD Ryzen 7 6800H with Radeon Graphics \\ \hline
					内存 & 16GB  \\ \hline
					硬盘空间 & 可用空间 1GB 以上 \\ \hline
					测试工具 & Qt Creator 自带调试工具、Windows任务管理器 \\ \hline
				\end{tabular}
				\caption{测试环境配置}
			\end{table}
		
		\subsection{测试方法}
			本次测试采用\textbf{黑盒测试}方法,主要测试策略包括:
			\begin{itemize}[leftmargin=2cm]
				\item \textbf{等价类划分}:针对输入数据进行有效等价类和无效等价类划分
				\item \textbf{边界值分析}:测试输入数据的边界条件
				\item \textbf{因果图法}:分析输入条件之间的逻辑关系
				\item \textbf{场景测试}:模拟用户实际使用场景进行端到端测试
				\item \textbf{性能测试}:监控系统运行时的资源占用情况
			\end{itemize}
	
	\newpage
	\section{操作界面测试}
	\subsection{账户登录测试}
		\subsubsection{测试目标}
			验证人员登录的正确性,包括用户登录、管理员登录。
		
		\subsubsection{测试用例设计}
			测试输入不同的 Username 和 Password,系统能否正确处理。
			
			\paragraph{用户登录测试}
			\begin{longtable}{|p{1cm}|p{3cm}|p{4.5cm}|p{3.5cm}|p{2cm}|}
				\hline
				\textbf{编号} & \textbf{测试项} & \textbf{输入数据} & \textbf{预期结果} & \textbf{测试结果} \\ \hline
				\endfirsthead
				\hline
				\textbf{编号} & \textbf{测试项} & \textbf{输入数据} & \textbf{预期结果} & \textbf{测试结果} \\ \hline
				\endhead
				TC-0.1 & 正确用户名和密码 & 用户名:1\newline 密码:1 & 登录成功,进入用户主页 &  登录成功,进入用户页面 \hline
				TC-0.2 & 用户名正确密码错误 & 用户名:1\newline 密码:2 & 提示密码错误 &  提示密码错误 \hline
				TC-0.3 & 用户名错误密码正确 & 用户名:wronguser\newline 密码:123456 & 提示用户不存在 &  提示用户不存在 \hline
				TC-0.4 & 用户名和密码都错误 & 用户名:wronguser\newline 密码:wrongpass & 提示用户不存在或密码错误 &  提示用户不存在 \hline
				TC-0.5 & 用户名为空 & 用户名:(空)\newline 密码:123456 & 提示用户名不能为空 &  提示用户不存在 \hline
				TC-0.6 & 密码为空 & 用户名:user1\newline 密码:(空) & 提示密码不能为空 &  提示密码错误 \hline
				TC-0.7 & 用户名和密码都为空 & 用户名:(空)\newline 密码:(空) & 提示用户名和密码不能为空 &  提示用户不存在 \hline
			TC-0.8 & 已锁定的用户登录 & 用户名:bob\newline 密码:qwerty & 提示账户已被锁定 &  提示账户已被锁定,请联系管理员解锁 \hline
			TC-0.9 & 锁定用户-空密码 & 用户名:bob\newline 密码:(空) & 提示密码不能为空 &  提示账户已被锁定,请联系管理员解锁 \hline
			TC-0.10 & 锁定用户-错误密码 & 用户名:bob\newline 密码:wrongpass & 提示密码错误 &  提示账户已被锁定,请联系管理员解锁 \hline
			TC-0.11 & 特殊字符用户名 & 用户名:user@\#\$\%\newline 密码:123456 & 处理特殊字符或提示格式错误 &  提示用户不存在 \hline
			TC-0.12 & 超长用户名 & 用户名:超过50个字符的用户名\newline 密码:123456 & 提示用户名过长或截断处理 &  提示用户不存在,前端页面脱标 \hline
			\caption{用户登录测试用例}
		\end{longtable}
			
			\paragraph{管理员登录测试}
			\begin{longtable}{|p{1cm}|p{3cm}|p{4.5cm}|p{3.5cm}|p{2cm}|}
				\hline
				\textbf{编号} & \textbf{测试项} & \textbf{输入数据} & \textbf{预期结果} & \textbf{测试结果} \\ \hline
				\endfirsthead
				\hline
				\textbf{编号} & \textbf{测试项} & \textbf{输入数据} & \textbf{预期结果} & \textbf{测试结果} \\ \hline
				\endhead
				TC-0.11 & 正确管理员账号密码 & 用户名:2\newline 密码:2 & 登录成功,进入管理员后台 &  登录成功,进入管理员后台 \hline
				TC-0.12 & 管理员名正确密码错误 & 用户名:2\newline 密码:wrongpass & 提示密码错误 &  提示密码错误 \hline
				TC-0.13 & 管理员名错误密码正确 & 用户名:wrongadmin\newline 密码:2 & 提示管理员不存在 &  提示账户不存在 \hline
				TC-0.14 & 管理员名密码都错误 & 用户名:wrongadmin\newline 密码:wrongpass & 提示管理员不存在或密码错误 &  提示账户不存在 \hline
				TC-0.15 & 管理员名为空 & 用户名:(空)\newline 密码:2 & 提示用户名不能为空 &  提示账户不存在 \hline
				TC-0.16 & 管理员密码为空 & 用户名:2\newline 密码:(空) & 提示密码不能为空 &  提示密码错误 \hline
			TC-0.17 & 已锁定的管理员登录 & 用户名:locked\_admin\newline 密码:admin123 & 提示账户已被锁定 &  提示账户已被锁定,请联系管理员解锁 \hline
			TC-0.18 & 锁定管理员-空密码 & 用户名:locked\_admin\newline 密码:(空) & 提示密码不能为空 &  提示账户已被锁定,请联系管理员解锁 \hline
			TC-0.19 & 锁定管理员-错误密码 & 用户名:locked\_admin\newline 密码:wrongpass & 提示密码错误 &  提示账户已被锁定,请联系管理员解锁 \hline
			TC-0.20 & 用普通用户账号登管理员 & 用户名:1\newline 密码:1 & 提示管理员不存在或权限不足 &  提示账户不存在 \hline
			\caption{管理员登录测试用例}
		\end{longtable}
		
		\subsubsection{测试结果总结}
			\paragraph{用户登录测试结果}
			共执行12个测试用例,其中:
			\begin{itemize}[leftmargin=2cm]
				\item \textbf{通过}: 8个 (TC-0.1, TC-0.2, TC-0.3, TC-0.4, TC-0.8的部分功能)
				\item \textbf{失败}: 4个 (TC-0.5, TC-0.6, TC-0.7, TC-0.9, TC-0.10验证顺序问题)
				\item \textbf{部分通过}: 2个 (TC-0.11, TC-0.12界面显示异常)
			\end{itemize}
			
			\paragraph{管理员登录测试结果}
			共执行10个测试用例,其中:
			\begin{itemize}[leftmargin=2cm]
				\item \textbf{通过}: 7个 (TC-0.11至TC-0.14, TC-0.17的部分功能)
				\item \textbf{失败}: 3个 (TC-0.15, TC-0.16, TC-0.18, TC-0.19验证顺序问题)
			\end{itemize}
			
		\subsubsection{问题记录}
			\begin{longtable}{|p{1.5cm}|p{1.5cm}|p{5cm}|p{5cm}|}
				\hline
				\textbf{问题编号} & \textbf{严重程度} & \textbf{问题描述} & \textbf{建议修复方案} \\ \hline
				\endfirsthead
				\hline
				\textbf{问题编号} & \textbf{严重程度} & \textbf{问题描述} & \textbf{建议修复方案} \\ \hline
				\endhead
				BUG-L01 & 中 & 用户名为空时,系统提示"用户不存在"而非"用户名不能为空",提示信息不够友好 & 在后端验证逻辑中,优先检查输入是否为空,如果为空则返回更友好的提示信息"用户名不能为空" \\ \hline
				BUG-L02 & 中 & 密码为空时,系统提示"密码错误"而非"密码不能为空",提示信息不够准确 & 在后端验证逻辑中,优先检查密码是否为空,如果为空则返回"密码不能为空" \\ \hline
				BUG-L03 & 中 & 锁定用户未输入密码或输入错误密码时,系统优先提示"账户已被锁定"而非检查密码,验证顺序不合理 & 调整验证顺序:先检查输入完整性(是否为空)→再检查账户存在性→再检查密码正确性→最后检查账户锁定状态 \\ \hline
				BUG-L04 & 低 & 超长用户名输入时,前端页面出现"脱标"现象,界面显示异常 & 前端添加输入长度限制,超过50个字符时进行截断或提示;优化CSS样式,防止超长文本导致布局错乱 \\ \hline
				BUG-L05 & 低 & 特殊字符用户名处理不够明确,未给出格式错误提示 & 增加用户名格式验证,对特殊字符进行过滤或提示用户名只能包含字母、数字和下划线 \\ \hline
				\caption{账户登录测试问题列表}
			\end{longtable}
			
			\paragraph{核心问题分析}
			经过测试发现,系统的\textbf{验证顺序存在设计缺陷}:
			\begin{enumerate}[leftmargin=2cm]
				\item \textbf{当前验证顺序}:系统先检查账户是否存在,再检查账户状态(是否锁定),最后才检查密码
				\item \textbf{导致的问题}:
					\begin{itemize}
						\item 当用户名为空或密码为空时,不会给出准确的空值提示(TC-0.5, TC-0.6, TC-0.7失败)
						\item 对于锁定的账户,即使密码为空或错误,也会先提示"账户已被锁定"(TC-0.9, TC-0.10, TC-0.18, TC-0.19失败)
						\item \textbf{典型案例}:用户名为"bob"(已锁定用户),密码为空时,系统提示"账户已被锁定"而非"密码不能为空"
						\item 这种设计虽然在安全性上有一定考虑(防止通过错误信息判断账户是否存在),但在用户体验上不够友好,且逻辑上不合理——应该先检查用户输入的完整性
					\end{itemize}
				\item \textbf{建议的验证顺序}:
					\begin{itemize}
						\item 第一步:检查用户名和密码是否为空(前端和后端都要检查)
						\item 第二步:检查账户是否存在
						\item 第三步:验证密码是否正确
						\item 第四步:检查账户是否被锁定(仅在密码正确的情况下才提示锁定)
					\end{itemize}
			\end{enumerate}
			
			\paragraph{安全性与用户体验的平衡}
			在实际应用中,过于详细的错误提示可能会带来安全隐患(如攻击者可以通过错误信息判断账户是否存在)。建议采用以下折中方案:
			\begin{itemize}[leftmargin=2cm]
				\item 对于空值输入:明确提示"用户名/密码不能为空"
				\item 对于用户名或密码错误:统一提示"用户名或密码错误"(不透露具体是哪个错误)
				\item 对于锁定账户:在确认账户存在且密码正确的前提下,再提示"账户已被锁定"
			\end{itemize}	
	\subsection{人员信息输入输出/编辑界面测试}
			\subsubsection{测试目标}
				验证人员信息输入、输出、编辑功能的正确性,包括用户注册、乘车人添加/修改等涉及人员信息的界面。
			
			\subsubsection{测试用例设计}
				\paragraph{控件输入限定测试}
				\begin{longtable}{|p{1cm}|p{3cm}|p{4cm}|p{3cm}|p{2cm}|}
					\hline
					\textbf{编号} & \textbf{测试项} & \textbf{输入数据} & \textbf{预期结果} & \textbf{测试结果} \\ \hline
					\endfirsthead
					\hline
					\textbf{编号} & \textbf{测试项} & \textbf{输入数据} & \textbf{预期结果} & \textbf{测试结果} \\ \hline
					\endhead
					TC-1.1 & 姓名长度限制 & 输入1个字符的姓名 & 提示姓名长度不符或允许输入 &  允许输入 \hline
					TC-1.2 & 姓名长度限制 & 输入超过20个字符的姓名 & 提示姓名过长或截断 &  允许输入 \hline
					TC-1.3 & 身份证号格式 & 输入15位数字 & 提示格式错误(应为18位) &  提示格式错误 \hline
					TC-1.4 & 身份证号格式 & 输入正确的18位身份证号 & 接受输入 &  接受输入 \hline
					TC-1.5 & 身份证号格式 & 输入包含字母的身份证号 & 提示格式错误 &  不允许输入字母 \hline
					TC-1.6 & 手机号格式 & 输入10位数字 & 提示格式错误(应为11位) &  提示格式错误 \hline
					TC-1.7 & 手机号格式 & 输入正确的11位手机号 & 接受输入 &  接受输入 \hline
					TC-1.8 & 用户名长度 & 输入空用户名 & 提示用户名不能为空 &  提示用户名不能为空 \hline
					TC-1.9 & 用户名长度 & 输入超长用户名(>50字符) & 提示用户名过长或截断 &  限制输入长度 \hline
					\caption{控件输入限定测试用例}
				\end{longtable}
				
				\paragraph{人员信息完整性测试}
				\begin{longtable}{|p{1cm}|p{3cm}|p{4cm}|p{3cm}|p{2cm}|}
					\hline
					\textbf{编号} & \textbf{测试项} & \textbf{输入数据} & \textbf{预期结果} & \textbf{测试结果} \\ \hline
					\endfirsthead
					\hline
					\textbf{编号} & \textbf{测试项} & \textbf{输入数据} & \textbf{预期结果} & \textbf{测试结果} \\ \hline
					\endhead
					TC-2.1 & 必填项缺失 & 姓名为空,其他正常 & 提示姓名不能为空 &  提示姓名不能为空 \hline
					TC-2.2 & 必填项缺失 & 身份证号为空 & 提示身份证号不能为空 &  提示身份证号不能为空 \hline
					TC-2.3 & 必填项缺失 & 手机号为空 & 提示手机号不能为空 &  手机号不能为空 \hline
					TC-2.4 & 完整信息提交 & 所有必填项均正确填写 & 成功保存/注册 &  成功保存/注册 \hline
					TC-2.5 & 信息修改 & 修改已存在的乘车人姓名 & 成功修改 &  成功修改 \hline
					TC-2.6 & 信息修改 & 修改已存在的手机号 & 成功修改 &  成功修改 \hline
					\caption{人员信息完整性测试用例}
				\end{longtable}
			
			\subsubsection{测试结果}
				% 在实际测试后填写
				
			\subsubsection{问题记录}
				% 记录测试中发现的问题
		
		\subsection{人员信息列表显示界面测试}
			\subsubsection{测试目标}
				验证列表控件能正确显示当前人员信息集合中的所有人员,并支持信息显示和排序功能。
			
			\subsubsection{测试用例设计}
				\begin{longtable}{|p{1cm}|p{3.5cm}|p{4cm}|p{3cm}|p{2cm}|}
					\hline
					\textbf{编号} & \textbf{测试项} & \textbf{测试操作} & \textbf{预期结果} & \textbf{测试结果} \\ \hline
					\endfirsthead
					\hline
					\textbf{编号} & \textbf{测试项} & \textbf{测试操作} & \textbf{预期结果} & \textbf{测试结果} \\ \hline
					\endhead
					TC-4.1 & 空集合显示 & 无任何人员信息时查看列表 & 显示空列表或提示无数据 &  \\ \hline
					TC-4.2 & 单个人员显示 & 仅有1个人员信息 & 正确显示该人员信息 &  \\ \hline
					TC-4.3 & 多个人员显示 & 有多个人员信息 & 正确显示所有人员信息 &  \\ \hline
					TC-4.4 & 信息完整性 & 检查显示的信息字段 & 姓名、身份证、手机号等完整显示 &  \\ \hline
					TC-4.5 & 列表排序 & 按姓名排序 & 列表按姓名顺序显示 &  \\ \hline
					TC-4.6 & 列表刷新 & 添加/删除人员后 & 列表自动刷新显示最新数据 &  \\ \hline
					TC-4.7 & 搜索过滤 & 输入关键字搜索 & 显示匹配的人员信息 &  \\ \hline
					\caption{人员信息列表显示测试用例}
				\end{longtable}
			
			\subsubsection{测试结果}
				% 在实际测试后填写
			
			\subsubsection{问题记录}
				% 记录测试中发现的问题
	
	\newpage
	\section{功能测试}
		\subsection{信息文件装载功能}
			\subsubsection{测试目标}
				验证系统启动时能够正确从文件中读取数据,包括用户信息、车次信息、订单信息等,并正确加载到内存中。
			
			\subsubsection{测试用例设计}
				\begin{longtable}{|p{1cm}|p{3cm}|p{4cm}|p{3.5cm}|p{2cm}|}
					\hline
					\textbf{编号} & \textbf{测试项} & \textbf{测试场景} & \textbf{预期结果} & \textbf{测试结果} \\ \hline
					\endfirsthead
					\hline
					\textbf{编号} & \textbf{测试项} & \textbf{测试场景} & \textbf{预期结果} & \textbf{测试结果} \\ \hline
					\endhead
					TC-5.1 & 正常文件加载 & 数据文件存在且数据记录>0 & 所有数据正确加载并在界面显示 &  \\ \hline
					TC-5.2 & 空文件加载 & 数据文件存在但记录为0 & 系统正常启动,显示空数据 &  \\ \hline
					TC-5.3 & 文件不存在 & 删除某个数据文件 & 系统提示文件缺失或创建新文件 &  \\ \hline
					TC-5.4 & 文件格式错误 & 修改文件内容为非法格式 & 系统提示格式错误或跳过错误数据 &  \\ \hline
					TC-5.5 & 数据一致性 & 加载后检查数据 & 文件中的数据与界面显示一致 &  \\ \hline
					TC-5.6 & 多文件加载 & 所有数据文件同时加载 & 所有模块数据正确加载 &  \\ \hline
					\caption{信息文件装载测试用例}
				\end{longtable}
			
			\subsubsection{测试结果}
				% 在实际测试后填写
			
			\subsubsection{问题记录}
				% 记录测试中发现的问题
		
		\subsection{信息添加功能}
			\subsubsection{测试目标}
				验证系统的信息添加功能,包括用户注册、乘车人添加、车次添加等,确保能正确处理唯一性约束(如身份证号不能重复)。
			
			\subsubsection{测试用例设计}
				\begin{longtable}{|p{1cm}|p{3cm}|p{4cm}|p{3.5cm}|p{2cm}|}
					\hline
					\textbf{编号} & \textbf{测试项} & \textbf{测试数据} & \textbf{预期结果} & \textbf{测试结果} \\ \hline
					\endfirsthead
					\hline
					\textbf{编号} & \textbf{测试项} & \textbf{测试数据} & \textbf{预期结果} & \textbf{测试结果} \\ \hline
					\endhead
					TC-6.1 & 添加唯一信息 & 添加唯一身份证号的乘车人 & 成功添加 &  \\ \hline
					TC-6.2 & 添加重复信息 & 添加已存在的身份证号 & 拒绝添加,提示已存在 &  \\ \hline
					TC-6.3 & 用户注册 & 注册新用户名 & 成功注册 &  \\ \hline
					TC-6.4 & 用户名重复 & 注册已存在的用户名 & 拒绝注册,提示用户名已存在 &  \\ \hline
					TC-6.5 & 车次添加 & 添加新车次号 & 成功添加车次 &  \\ \hline
					TC-6.6 & 车次重复 & 添加已存在的车次号 & 拒绝添加,提示车次已存在 &  \\ \hline
					TC-6.7 & 数据持久化 & 添加后重启系统 & 新添加的数据仍然存在 &  \\ \hline
					\caption{信息添加功能测试用例}
				\end{longtable}
			
			\subsubsection{测试结果}
				% 在实际测试后填写
			
			\subsubsection{问题记录}
				% 记录测试中发现的问题
		
		\subsection{信息查询功能}
			\subsubsection{测试目标}
				验证各种查询功能的正确性,包括余票查询、订单查询、乘车人查询等。
			
			\subsubsection{测试用例设计}
				\begin{longtable}{|p{1cm}|p{3cm}|p{4cm}|p{3.5cm}|p{2cm}|}
					\hline
					\textbf{编号} & \textbf{测试项} & \textbf{测试数据} & \textbf{预期结果} & \textbf{测试结果} \\ \hline
					\endfirsthead
					\hline
					\textbf{编号} & \textbf{测试项} & \textbf{测试数据} & \textbf{预期结果} & \textbf{测试结果} \\ \hline
					\endhead
					TC-7.1 & 余票查询-有票 & 查询有余票的车次 & 返回可预订车次列表 &  \\ \hline
					TC-7.2 & 余票查询-无票 & 查询无余票的车次 & 显示无票或售罄 &  \\ \hline
					TC-7.3 & 余票查询-不存在 & 查询不存在的线路 & 提示无该线路车次 &  \\ \hline
					TC-7.4 & 订单查询 & 查询用户订单 & 返回该用户所有订单 &  \\ \hline
					TC-7.5 & 订单搜索 & 按订单号搜索 & 返回匹配的订单 &  \\ \hline
					TC-7.6 & 乘车人查询 & 查询用户乘车人 & 返回该用户所有乘车人 &  \\ \hline
					TC-7.7 & 多条件筛选 & 使用多个筛选条件 & 返回符合所有条件的结果 &  \\ \hline
					TC-7.8 & 日期查询 & 选择不同日期查询 & 返回对应日期的车次 &  \\ \hline
					\caption{信息查询功能测试用例}
				\end{longtable}
			
			\subsubsection{测试结果}
				% 在实际测试后填写
			
			\subsubsection{问题记录}
				% 记录测试中发现的问题
		
		\subsection{信息删除功能}
			\subsubsection{测试目标}
				验证信息删除功能,包括乘车人删除、订单取消、用户注销等,确保级联删除正确执行。
			
			\subsubsection{测试用例设计}
				\begin{longtable}{|p{1cm}|p{3cm}|p{4cm}|p{3.5cm}|p{2cm}|}
					\hline
					\textbf{编号} & \textbf{测试项} & \textbf{测试操作} & \textbf{预期结果} & \textbf{测试结果} \\ \hline
					\endfirsthead
					\hline
					\textbf{编号} & \textbf{测试项} & \textbf{测试操作} & \textbf{预期结果} & \textbf{测试结果} \\ \hline
					\endhead
					TC-8.1 & 删除乘车人-无订单 & 删除无待乘坐订单的乘车人 & 成功删除 &  \\ \hline
					TC-8.2 & 删除乘车人-有订单 & 删除有待乘坐订单的乘车人 & 拒绝删除,提示有关联订单 &  \\ \hline
					TC-8.3 & 订单取消 & 取消待乘坐订单 & 成功取消,释放座位 &  \\ \hline
					TC-8.4 & 订单取消-已完成 & 取消已完成订单 & 拒绝取消或给出提示 &  \\ \hline
					TC-8.5 & 用户注销 & 注销用户账号 & 级联删除所有订单和乘车人 &  \\ \hline
					TC-8.6 & 车次删除-无订单 & 删除无待乘坐订单的车次 & 成功删除 &  \\ \hline
					TC-8.7 & 车次删除-有订单 & 删除有待乘坐订单的车次 & 拒绝删除,提示有关联订单 &  \\ \hline
					TC-8.8 & 删除后数据一致性 & 删除后重启系统 & 已删除数据不再出现 &  \\ \hline
					\caption{信息删除功能测试用例}
				\end{longtable}
			
			\subsubsection{测试结果}
				% 在实际测试后填写
			
			\subsubsection{问题记录}
				% 记录测试中发现的问题
		
		\subsection{信息修改功能}
			\subsubsection{测试目标}
				验证信息修改功能,包括个人信息修改、乘车人信息修改、车次信息修改等。
			
			\subsubsection{测试用例设计}
				\begin{longtable}{|p{1cm}|p{3cm}|p{4cm}|p{3.5cm}|p{2cm}|}
					\hline
					\textbf{编号} & \textbf{测试项} & \textbf{测试操作} & \textbf{预期结果} & \textbf{测试结果} \\ \hline
					\endfirsthead
					\hline
					\textbf{编号} & \textbf{测试项} & \textbf{测试操作} & \textbf{预期结果} & \textbf{测试结果} \\ \hline
					\endhead
					TC-9.1 & 修改个人信息 & 修改姓名、手机号 & 成功修改并保存 &  \\ \hline
					TC-9.2 & 修改乘车人-无订单 & 修改无待乘坐订单的乘车人 & 成功修改 &  \\ \hline
					TC-9.3 & 修改乘车人-有订单 & 修改有待乘坐订单的乘车人 & 拒绝修改或提示 &  \\ \hline
					TC-9.4 & 修改车次时刻表 & 修改停靠站信息 & 成功修改时刻表 &  \\ \hline
					TC-9.5 & 修改座位模板 & 修改车厢数和座位布局 & 成功修改座位模板 &  \\ \hline
					TC-9.6 & 修改为重复数据 & 修改为已存在的身份证号 & 拒绝修改,提示重复 &  \\ \hline
					TC-9.7 & 修改后数据一致性 & 修改后重启系统 & 修改的数据正确保存 &  \\ \hline
					\caption{信息修改功能测试用例}
				\end{longtable}
			
			\subsubsection{测试结果}
				% 在实际测试后填写
			
			\subsubsection{问题记录}
				% 记录测试中发现的问题
		
		\subsection{订单管理功能}
			\subsubsection{测试目标}
				验证订单创建、改签、退票等核心业务功能。
			
			\subsubsection{测试用例设计}
				\begin{longtable}{|p{1cm}|p{3cm}|p{4cm}|p{3.5cm}|p{2cm}|}
					\hline
					\textbf{编号} & \textbf{测试项} & \textbf{测试操作} & \textbf{预期结果} & \textbf{测试结果} \\ \hline
					\endfirsthead
					\hline
					\textbf{编号} & \textbf{测试项} & \textbf{测试操作} & \textbf{预期结果} & \textbf{测试结果} \\ \hline
					\endhead
					TC-10.1 & 创建订单-有余票 & 选择有余票的车次订票 & 成功创建订单,分配座位 &  \\ \hline
					TC-10.2 & 创建订单-无余票 & 选择无余票的车次订票 & 拒绝订票,提示无票 &  \\ \hline
					TC-10.3 & 创建订单-乘车人冲突 & 选择已有待乘坐订单的乘车人 & 拒绝订票,提示时间冲突 &  \\ \hline
					TC-10.4 & 订单改签 & 改签到其他车次 & 成功改签,原订单取消 &  \\ \hline
					TC-10.5 & 订单退票 & 退票操作 & 成功退票,释放座位 &  \\ \hline
					TC-10.6 & 订单查看 & 查看订单详情 & 显示完整订单信息 &  \\ \hline
					TC-10.7 & 座位分配 & 创建订单时 & 自动分配座位号 &  \\ \hline
					TC-10.8 & 订单编号生成 & 创建订单时 & 生成唯一订单编号 &  \\ \hline
					\caption{订单管理功能测试用例}
				\end{longtable}
			
			\subsubsection{测试结果}
				% 在实际测试后填写
			
			\subsubsection{问题记录}
				% 记录测试中发现的问题
		
		\subsection{信息排序功能}
			\subsubsection{测试目标}
				验证余票查询结果的排序功能,包括按时间、价格等排序。
			
			\subsubsection{测试用例设计}
				\begin{longtable}{|p{1cm}|p{3cm}|p{4cm}|p{3.5cm}|p{2cm}|}
					\hline
					\textbf{编号} & \textbf{测试项} & \textbf{测试操作} & \textbf{预期结果} & \textbf{测试结果} \\ \hline
					\endfirsthead
					\hline
					\textbf{编号} & \textbf{测试项} & \textbf{测试操作} & \textbf{预期结果} & \textbf{测试结果} \\ \hline
					\endhead
					TC-11.1 & 按出发时间排序 & 点击时间排序按钮 & 车次按出发时间升序/降序 &  \\ \hline
					TC-11.2 & 按价格排序 & 点击价格排序按钮 & 车次按价格升序/降序 &  \\ \hline
					TC-11.3 & 按耗时排序 & 点击耗时排序按钮 & 车次按总耗时升序/降序 &  \\ \hline
					TC-11.4 & 默认排序 & 无排序操作 & 按默认规则显示 &  \\ \hline
					TC-11.5 & 空结果排序 & 对空查询结果排序 & 不报错,显示空列表 &  \\ \hline
					\caption{信息排序功能测试用例}
				\end{longtable}
			
			\subsubsection{测试结果}
				% 在实际测试后填写
			
			\subsubsection{问题记录}
				% 记录测试中发现的问题
		
		\subsection{清空信息功能}
			\subsubsection{测试目标}
				验证清空查询历史等清空操作的正确性。
			
			\subsubsection{测试用例设计}
				\begin{longtable}{|p{1cm}|p{3cm}|p{4cm}|p{3.5cm}|p{2cm}|}
					\hline
					\textbf{编号} & \textbf{测试项} & \textbf{测试操作} & \textbf{预期结果} & \textbf{测试结果} \\ \hline
					\endfirsthead
					\hline
					\textbf{编号} & \textbf{测试项} & \textbf{测试操作} & \textbf{预期结果} & \textbf{测试结果} \\ \hline
					\endhead
					TC-12.1 & 清空查询历史 & 点击清空历史按钮 & 所有历史记录被清除 &  \\ \hline
					TC-12.2 & 清空空历史 & 在无历史记录时清空 & 不报错,保持空状态 &  \\ \hline
					TC-12.3 & 清空后再添加 & 清空后重新查询 & 新历史正常添加 &  \\ \hline
					\caption{清空信息功能测试用例}
				\end{longtable}
			
			\subsubsection{测试结果}
				% 在实际测试后填写
			
			\subsubsection{问题记录}
				% 记录测试中发现的问题
		
		\subsection{信息文件保存功能}
			\subsubsection{测试目标}
				验证系统退出时能够正确保存所有修改的数据,并在必要时创建备份文件。
			
			\subsubsection{测试用例设计}
				\begin{longtable}{|p{1cm}|p{3cm}|p{4cm}|p{3.5cm}|p{2cm}|}
					\hline
					\textbf{编号} & \textbf{测试项} & \textbf{测试操作} & \textbf{预期结果} & \textbf{测试结果} \\ \hline
					\endfirsthead
					\hline
					\textbf{编号} & \textbf{测试项} & \textbf{测试操作} & \textbf{预期结果} & \textbf{测试结果} \\ \hline
					\endhead
					TC-13.1 & 正常退出保存 & 进行操作后正常退出 & 数据正确保存到文件 &  \\ \hline
					TC-13.2 & 备份文件创建 & 首次退出时 & 创建原有信息的备份文件 &  \\ \hline
					TC-13.3 & 数据完整性 & 保存后重新加载 & 所有数据与退出前一致 &  \\ \hline
					TC-13.4 & 文件格式 & 检查保存的文件 & 文件格式正确,可读取 &  \\ \hline
					TC-13.5 & 异常退出 & 强制关闭程序 & 部分数据可能丢失(可接受) &  \\ \hline
					\caption{信息文件保存功能测试用例}
				\end{longtable}
			
			\subsubsection{测试结果}
				% 在实际测试后填写
			
			\subsubsection{问题记录}
				% 记录测试中发现的问题
	
	\newpage
	\section{内存泄漏测试}
		\subsection{测试目标}
			在Debug模式下运行程序,执行各功能操作后正常退出,检查系统运行过程中是否存在内存泄漏问题。
		
		\subsection{测试方法}
			\begin{enumerate}[leftmargin=2cm]
				\item 使用Qt Creator的调试工具或Windows任务管理器监控程序内存使用情况
				\item 在Debug模式下启动程序
				\item 依次执行以下操作:
					\begin{itemize}
						\item 用户登录/注册
						\item 余票查询(多次查询不同线路)
						\item 创建订单
						\item 修改乘车人信息
						\item 查看订单列表
						\item 改签/退票
						\item 管理员登录
						\item 车次管理操作
						\item 用户管理操作
					\end{itemize}
				\item 记录每步操作后的内存使用情况
				\item 正常退出程序
				\item 检查是否有内存未释放的警告信息
			\end{enumerate}
		
		\subsection{测试数据记录}
			\begin{table}[H]
				\centering
				\begin{tabular}{|l|c|c|}
					\hline
					\textbf{操作步骤} & \textbf{内存使用(MB)} & \textbf{备注} \\ \hline
					程序启动 &  &  \\ \hline
					用户登录 &  &  \\ \hline
					余票查询(第1次) &  &  \\ \hline
					余票查询(第5次) &  &  \\ \hline
					余票查询(第10次) &  &  \\ \hline
					创建订单 &  &  \\ \hline
					订单查询 &  &  \\ \hline
					改签操作 &  &  \\ \hline
					退票操作 &  &  \\ \hline
					管理员操作 &  &  \\ \hline
					程序退出前 &  &  \\ \hline
				\end{tabular}
				\caption{内存使用情况记录表}
			\end{table}
		
		\subsection{测试结果}
			% 在实际测试后填写
			
		\subsection{问题记录}
			% 记录发现的内存泄漏问题
	
	\newpage
	\section{CPU占用测试}
		\subsection{测试目标}
			在系统"任务管理器"的监控下运行程序,执行各功能操作时,查看CPU占用状态的变化和瞬间峰值。
		
		\subsection{测试方法}
			\begin{enumerate}[leftmargin=2cm]
				\item 打开Windows任务管理器,切换到"性能"或"详细信息"标签
				\item 启动高铁订票管理系统程序
				\item 依次执行以下操作并记录CPU占用:
					\begin{itemize}
						\item 程序空闲状态
						\item 用户登录
						\item 余票查询(单次)
						\item 连续多次查询(压力测试)
						\item 创建订单
						\item 加载大量订单列表
						\item 修改车次时刻表
						\item 修改座位模板
						\item 数据保存
					\end{itemize}
				\item 记录每个操作的CPU占用峰值和平均值
			\end{enumerate}
		
		\subsection{测试数据记录}
			\begin{table}[H]
				\centering
				\begin{tabular}{|l|c|c|c|}
					\hline
					\textbf{操作步骤} & \textbf{CPU峰值(\%)} & \textbf{CPU平均(\%)} & \textbf{备注} \\ \hline
					程序空闲 &  &  &  \\ \hline
					用户登录 &  &  &  \\ \hline
					余票查询(单次) &  &  &  \\ \hline
					连续查询(10次) &  &  &  \\ \hline
					创建订单 &  &  &  \\ \hline
					加载100个订单 &  &  &  \\ \hline
					修改时刻表 &  &  &  \\ \hline
					修改座位模板 &  &  &  \\ \hline
					数据保存 &  &  &  \\ \hline
				\end{tabular}
				\caption{CPU占用情况记录表}
			\end{table}
		
		\subsection{性能评估标准}
			\begin{itemize}[leftmargin=2cm]
				\item 空闲状态:CPU占用应 < 5\%
				\item 普通操作:CPU占用峰值应 < 30\%
				\item 复杂操作(如数据加载、排序):CPU占用峰值应 < 60\%
				\item 瞬时峰值后应快速下降,不应持续高占用
			\end{itemize}
		
		\subsection{测试结果}
			% 在实际测试后填写
			
		\subsection{问题记录}
			% 记录CPU占用异常的问题
	
	\newpage
	\section{缺陷统计与分析}
		\subsection{缺陷分类}
			\begin{table}[H]
				\centering
				\begin{tabular}{|l|c|p{8cm}|}
					\hline
					\textbf{缺陷等级} & \textbf{数量} & \textbf{定义} \\ \hline
					严重(Blocker) &  & 导致系统崩溃或核心功能无法使用 \\ \hline
					高(Critical) &  & 重要功能异常,有变通方案 \\ \hline
					中(Major) &  & 非核心功能异常,影响用户体验 \\ \hline
					低(Minor) &  & 界面美观、提示信息等小问题 \\ \hline
					建议(Suggestion) &  & 功能改进建议 \\ \hline
					\textbf{总计} &  &  \\ \hline
				\end{tabular}
				\caption{缺陷等级统计表}
			\end{table}
		
		\subsection{缺陷列表}
			\begin{longtable}{|p{1cm}|p{1.5cm}|p{3cm}|p{4cm}|p{3cm}|}
				\hline
				\textbf{编号} & \textbf{等级} & \textbf{缺陷描述} & \textbf{复现步骤} & \textbf{建议修复方案} \\ \hline
				\endfirsthead
				\hline
				\textbf{编号} & \textbf{等级} & \textbf{缺陷描述} & \textbf{复现步骤} & \textbf{建议修复方案} \\ \hline
				\endhead
				% 示例缺陷记录,实际测试时填写
				% BUG-1 & 高 & 创建订单时未检查余票 & 1.查询余票 2.直接创建订单 3.未判断余票数量 & 添加余票数量校验 \\ \hline
				\caption{缺陷详细列表}
			\end{longtable}
		
		\subsection{缺陷分布分析}
			% 可以添加图表展示缺陷在各模块的分布情况
			\begin{table}[H]
				\centering
				\begin{tabular}{|l|c|c|}
					\hline
					\textbf{功能模块} & \textbf{缺陷数量} & \textbf{占比(\%)} \\ \hline
					用户登录/注册 &  &  \\ \hline
					余票查询 &  &  \\ \hline
					订单管理 &  &  \\ \hline
					乘车人管理 &  &  \\ \hline
					车次管理 &  &  \\ \hline
					数据持久化 &  &  \\ \hline
					界面交互 &  &  \\ \hline
					其他 &  &  \\ \hline
					\textbf{总计} &  & 100 \\ \hline
				\end{tabular}
				\caption{缺陷模块分布表}
			\end{table}
	
	\newpage
	\section{测试总结}
		\subsection{测试完成情况}
			\begin{table}[H]
				\centering
				\begin{tabular}{|l|c|c|c|}
					\hline
					\textbf{测试类型} & \textbf{计划用例数} & \textbf{执行用例数} & \textbf{通过率(\%)} \\ \hline
					操作界面测试 &  &  &  \\ \hline
					功能测试 &  &  &  \\ \hline
					性能测试 &  &  &  \\ \hline
					\textbf{总计} &  &  &  \\ \hline
				\end{tabular}
				\caption{测试执行情况统计}
			\end{table}
		
		\subsection{主要发现}
			% 总结测试中发现的主要问题
			\subsubsection{优点}
				\begin{itemize}[leftmargin=2cm]
					\item 
					\item 
					\item 
				\end{itemize}
			
			\subsubsection{待改进项}
				\begin{itemize}[leftmargin=2cm]
					\item 
					\item 
					\item 
				\end{itemize}
		
		\subsection{风险评估}
			% 评估当前版本的质量风险
			\begin{table}[H]
				\centering
				\begin{tabular}{|l|c|p{7cm}|}
					\hline
					\textbf{风险项} & \textbf{等级} & \textbf{说明} \\ \hline
					数据丢失风险 &  &  \\ \hline
					性能问题 &  &  \\ \hline
					用户体验 &  &  \\ \hline
					数据一致性 &  &  \\ \hline
				\end{tabular}
				\caption{质量风险评估表}
			\end{table}
		
		\subsection{测试结论}
			% 给出最终的测试结论和发布建议
			
		\subsection{改进建议}
			\subsubsection{功能改进}
				\begin{enumerate}[leftmargin=2cm]
					\item 
					\item 
					\item 
				\end{enumerate}
			
			\subsubsection{性能优化}
				\begin{enumerate}[leftmargin=2cm]
					\item 
					\item 
					\item 
				\end{enumerate}
			
			\subsubsection{用户体验优化}
				\begin{enumerate}[leftmargin=2cm]
					\item 
					\item 
					\item 
				\end{enumerate}
	
	\newpage
	\section{附录}
		\subsection{测试环境配置详情}
			% 详细的测试环境配置信息
			
		\subsection{测试数据说明}
			% 测试使用的数据说明
			
		\subsection{参考文档}
			\begin{itemize}[leftmargin=2cm]
				\item 《高铁订票管理系统开发文档》
				\item 《软件测试理论与实践》
				\item Qt官方文档
			\end{itemize}
		
		\subsection{术语表}
			\begin{table}[H]
				\centering
				\begin{tabular}{|l|p{10cm}|}
					\hline
					\textbf{术语} & \textbf{说明} \\ \hline
					黑盒测试 & 不考虑程序内部结构,仅从功能需求角度进行的测试 \\ \hline
					等价类划分 & 将输入数据划分为若干等价类,从中选取代表性数据进行测试 \\ \hline
					边界值分析 & 针对输入数据的边界值进行测试 \\ \hline
					级联删除 & 删除主记录时,自动删除所有相关联的从记录 \\ \hline
					内存泄漏 & 程序运行过程中动态分配的内存未正确释放 \\ \hline
				\end{tabular}
				\caption{术语表}
			\end{table}

\end{document}
